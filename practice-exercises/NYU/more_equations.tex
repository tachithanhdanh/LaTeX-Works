\documentclass[a4paper,12pt,notitlepage]{article}
\author{Tạ Chí Thành Danh}
\date{Ngày 06 tháng 05 năm 2022}
\title{Bài tập thực hành viết phương trình trong \LaTeX}
\usepackage[left=3cm,right=3cm,top=3cm,bottom=3cm]{geometry} % Định dạng khoảng cách lề giấy
\usepackage{amsmath} % gather* environment
\usepackage{commath} 
% https://tex.stackexchange.com/questions/412439/is-there-a-short-hand-command-to-write-derivatives
% There are several packages that display derivatives symbol elegantly
\begin{document}
\pagenumbering{gobble} 
% https://tex.stackexchange.com/questions/7355/how-to-suppress-page-number
\begin{gather*}
	% https://tex.stackexchange.com/questions/32824/show-inline-math-as-if-it-were-display-math-and-vice-versa
	% Displaying limits above and below integral sign 
	% https://tex.stackexchange.com/questions/19185/which-one-should-be-used-in-limit-rightarrow-or-to: Damn, \to and \rightarrow are the same thing.
	\int\limits_{a}^{b} x\dif x = \left.\dfrac{x^2}{2}\right|_a^b \\
	\iiint\limits_V f(x,y,z) \dif V = F \\
	\dod{x}{y} = x^\prime = \lim_{h \to 0} \dfrac{f(x+h)-f(x)}{h} \\
	\envert{x} = 
	\begin{cases}
		-x, & \text{if } x<0 \\
		x, & \text{if } x \geq 0
	\end{cases} \\
	F(x) = A_0 + \sum_{n=1}^N \intcc{A_n \cos \intoo{\dfrac{2\pi nx}{P}} + B_n \sin \intoo{\dfrac{2\pi nx}{P}}} \\
	\sum_n \dfrac{1}{n^s} = \prod_{p} \dfrac{1}{1-\tfrac{1}{p^n}} \\
	m\ddot{x} + c\dot{x} + kx = F_0\sin(2\pi ft)
	% https://tex.stackexchange.com/questions/152951/how-to-write-two-dot-above-a-letter#:~:text=Use%20%24%5Cdot%20x%24%20for,ddddot%20x%20for%20the%20fourth. 
\end{gather*}
\begin{align*}
	f(x) \quad &= \quad x^2 + 3x + 5x^2 + 5 + 6x \\
	&= \quad 6x^2 + 9x + 8 \\
	&= \quad x(6x+9) + 8 
\end{align*}
\begin{gather*}
	X = \dfrac{F_0}{k} \dfrac{1}{\sqrt{(1-r^2)^2+(2\zeta r)^2}} \\ \\
	G_{\mu \nu} \equiv R_{\mu \nu} - \dfrac{1}{2} Rg_{\mu \nu} = \dfrac{8\pi G}{c^4} T_{\mu \nu} \\ \\
	\mathrm{6CO_2 + 6H_2O} \rightarrow \mathrm{C_6H_{12}O_6 + 6O_2} \\ \\ 
	\mathrm{SO_4^{2-} + Ba^{2+}} \rightarrow \mathrm{BaSO_4} \\
	\begin{pmatrix}
		a_{11} & a_{12} & \cdots & a_{1n} \\
		a_{21} & a_{22} & \cdots & a_{2n} \\
		\vdots & \vdots & \ddots & \vdots \\
		a_{n1} & a_{n2} & \cdots & a_{nn} \\
	\end{pmatrix}
	\begin{pmatrix}
		v_1 \\ v_2 \\ \vdots \\ v_n
	\end{pmatrix}
	=
	\begin{pmatrix}
		w_1 \\ w_2 \\ \vdots \\ w_n
	\end{pmatrix} \\
	\dpd{\mathbf{u}}{t} + (\mathbf{u} \cdot \nabla)\mathbf{u} - \nu \nabla^2(\mathbf{u}) = -\nabla \mathbf{h} \\
	\alpha A \beta B \gamma \Gamma \delta \Delta \pi \Pi \omega \Omega 
\end{gather*}
\end{document}
% about \intcc, \intoo, \dod and \dpd, read here: https://mirror.kku.ac.th/CTAN/macros/latex/contrib/commath/commath.pdf