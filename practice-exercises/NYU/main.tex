\documentclass[a4paper,notitlepage]{article}
\title{Xin chào thế giới!}
\author{Tạ Chí Thành Danh}
\date{\today}
\usepackage[utf8]{vietnam} % Unicode tiếng Việt
\usepackage{amsmath}
\usepackage{graphicx} % Chèn hình ảnh bên ngoài vào
\usepackage[colorlinks]{hyperref}
\hypersetup{colorlinks = true, linkcolor=blue}
% https://tex.stackexchange.com/questions/50747/options-for-appearance-of-links-in-hyperref
\begin{document}
	\maketitle
	\section{Bắt đầu}
	\textbf{Xin chào thế giới!} Hôm nay tôi học về \LaTeX. \LaTeX{} là một chương trình tuyệt vời để biên soạn toán học. Tôi có thể viết phương trình toán học trong đoạn văn bản chẳng hạn như $a^2 + b^2 = c^2$. Tôi cũng có thể cho các phương trình có chỗ đứng riêng của nó:
	\begin{equation}
		\gamma^2 + \theta^2 = \omega^2
	\end{equation}
	``Phương trình Maxwell'' được đặt tên theo nhà vật lý học James Clark Maxwell và được viết như sau:
	\begin{align} 
		% https://latex-programming.fandom.com/wiki/Align_(LaTeX_environment)
		% https://support.authorea.com/en-us/article/how-do-i-reference-my-latex-tables-or-equations-brkb5m/#:~:text=To%20reference%20a%20LaTeX%20table,a%20eqn%3A%20prefix%20for%20equations.&text=Notice%20the%20%5Clabel%7Btab%3A,somelabel%7D%20inside%20the%20%5Ccaption%20.&text=Notice%20the%20%5Clabel%7Beqn%3Asomelabel%7D.
		% https://tex.stackexchange.com/questions/74353/what-commands-are-there-for-horizontal-spacing 
		% Different spacing characters in LaTeX
		\vec\nabla \cdot \vec E \enspace &= \enspace \dfrac{\rho}{\epsilon_0} && \text{Định luật Gauss} \label{eqn:2} \\
		\vec\nabla \cdot \vec B \enspace &= \enspace 0 && \text{Định luật Gauss về Từ trường} \label{eqn:3} \\
		\vec\nabla \times \vec E \enspace &= \enspace -\dfrac{\partial \vec B}{\partial t} && \text{Định luật Faraday về cảm ứng điện từ} \label{eqn:4} \\
		\vec\nabla \times \vec B \enspace &= \enspace \mu_0 \left(\epsilon_0 \dfrac{\partial \vec E}{\partial t} + \vec J\right) && \text{Định luật Ampère về dòng điện toàn phần} \label{eqn:5}
	\end{align}
	Các phương trình \ref{eqn:2}, \ref{eqn:3}, \ref{eqn:4}, và \ref{eqn:5} là một trong những phương trình quan trọng nhất trong Vật lý.
	\section{Vậy còn Phương trình Ma trận thì sao?}
	% https://latex-programming.fandom.com/wiki/Matrix_(LaTeX_environment)
	% https://latex-programming.fandom.com/wiki/Array_(LaTeX_environment)
	\begin{gather*}
	\begin{pmatrix}
		a_{11} & a_{12} & \cdots & a_{1n} \\
		a_{21} & a_{22} & \cdots & a_{2n} \\
		\vdots & \vdots & \ddots & \vdots \\
		a_{n1} & a_{m2} & \cdots & a_{nn}
	\end{pmatrix}
	\begin{bmatrix}
		v_1 \\ v_2 \\ \vdots \\ v_n 
	\end{bmatrix}
	=
	\begin{array}{c}
		w_1 \\ w_2 \\ \vdots \\ w_n
	\end{array}
	\end{gather*}
	\newpage
	\section{Bảng biểu và Hình ảnh}
	\begin{table}[h] 
		% https://tex.stackexchange.com/questions/8652/what-does-t-and-ht-mean
		% https://www.overleaf.com/learn/latex/Questions/How_can_I_get_my_table_or_figure_to_stay_where_they_are%2C_instead_of_going_to_the_next_page%3F
		% Không chèn [h] vô thì cái bảng sẽ nằm trên section luôn
		% https://www.overleaf.com/learn/latex/Questions/How_do_I_add_a_caption_to_a_table%3F
		\caption{Đây là một bảng cho thấy cách tạo nhiều dòng với nhiều kiểu căn lề khác nhau}
		\begin{center}
		\begin{tabular}{|l||c|c|r|}
				\hline
				$x$ & $1$ & $2$ & $3$ \\ 
				\hline
				$f(x)$ & $4$ & $8$ & $12$ \\ 
				%\hline
				f(x) & $4$ & $8$ & $12$ \\ 
				\hline
		\end{tabular}
		\end{center}
	\end{table}
	\begin{figure}[h]
		\begin{center}
			\includegraphics[scale=0.35]{logo_HCMUS.jpg}
		\end{center}
		\caption{Logo Trường Đại học Khoa học Tự nhiên, Đại học Quốc gia TP.HCM}
	\end{figure}
\end{document}
