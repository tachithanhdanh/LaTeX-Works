\documentclass[a4paper,12pt,notitlepage]{article}
\title{Bài tập thực hành \LaTeX}
\author{Tạ Chí Thành Danh}
\date{\TeX ed vào ngày 10, 11 và 12 tháng 4 năm 2022}
\usepackage[utf8]{vietnam} % Unicode tiếng Việt
\usepackage{amsmath}
\usepackage{amssymb} % \rightleftharpoons
\usepackage{mathtools} % \stackrel
\usepackage{esint} % \oiint
\usepackage{boldline} % make hline thicker
\usepackage{setspace}
\usepackage{lipsum}
\usepackage{xifthen}% provides \isempty test
%\usepackage{cancel} % cancel for striking through text
%\usepackage{soul} % \st strikethrough text
\usepackage[normalem]{ulem} % \sout
\usepackage[left=2cm,right=2cm,top=2cm,bottom=2cm]{geometry} % Định dạng khoảng cách lề giấy
\usepackage{hyperref}
\hypersetup{colorlinks, urlcolor=blue}
%\renewcommand{\baselinestretch}{0.5} 
\newcommand{\f}[2]{\dfrac{#1}{#2}}
\newcommand{\ud}{\,\mathrm{d}}
% https://tex.stackexchange.com/questions/117358/newcommand-argument-confusion
\newcommand{\mathset}[2][]{ % https://stackoverflow.com/questions/2144176/latex-newcommand-default-argument-is-empty
	% https://tex.stackexchange.com/questions/1232/difference-between-big-and-bigl
	% https://tex.stackexchange.com/questions/2607/spacing-around-left-and-right
	% https://tex.stackexchange.com/questions/1454/what-is-the-correct-way-to-do-delimiters
	% https://tex.stackexchange.com/questions/9091/what-is-the-right-way-to-use-the-spacing-command
	% https://tex.stackexchange.com/questions/41476/lengths-and-when-to-use-them/41484#41484
	% Negative space
	\ifthenelse{\isempty{#2}}
	{
		\emptyset
	}
	{
	\ifthenelse{\isempty{#1}}%
	{
	\left\{\!\!\!\begin{array}{c} % if #1 is empty
		#2
	\end{array}\!\!\!\right\}
	}
	{
	\left\{\!\!\!\begin{array}{c V{1.1} c} % if #1 is not empty
		#1 & #2
	\end{array}\!\!\!\right\}
	}
	}
}
\begin{document}
	\maketitle
	\begin{abstract}
		Đây là phần luyện tập của tôi nhằm củng cố thêm kỹ năng viết các công thức, phương trình toán học bằng \LaTeXe.
		Dành cho những ai cũng muốn luyện tập thì đây là \href{https://web.mit.edu/~jgross/Public/latex/exercises.pdf}{đường dẫn tới file bài tập}.
	\end{abstract}
	\section{Dễ}
	\begin{equation}
		\text{Please type me! The quick brown fox jumps over the lazy dog.}
	\end{equation}
	% https://tex.stackexchange.com/questions/2929/how-to-increase-the-spacing-between-equations-in-gather
	% https://tex.stackexchange.com/questions/277234/gather-vs-aligned-centered-equation-with-one-number
	% https://www.overleaf.com/learn/latex/Aligning_equations_with_amsmath
	\openup 2ex
	\begin{gather}
		e^{i \pi} + 1 = 0 \\
		e^{i \theta} = \cos \theta + i \sin \theta \\
		G_{\mu \nu} + \Lambda g_{\mu \nu} = \f{8\pi G}{c^4} T_{\mu \nu} \\
		x = \f{-b \pm \sqrt{b^2 - 4ac}}{2a} \\
		\vec{L} = \vec{r} \times \vec{p} \\
		\sqrt[3]{2} \\
		(x + y)^n = \sum_{r=0}^{n} \binom{n}{r} x^r y^{n-r} \\
		\sqrt{\f{a_1^2 + \cdots + a_n^2}{n}} \geq \f{a_1 + \cdots + a_n}{n} \geq \sqrt[n]{a_1 \cdots a_n} \geq \f{n}{\tfrac{1}{a_1} + \cdots + \tfrac{1}{a_n}} \\
		| \langle x, y \rangle | ^2 \leq \langle x, x \rangle \cdot \langle y, y \rangle %\\ \nonumber \\
	\end{gather}
	\begin{align}
	& \mathrm{A}1: \varphi \longrightarrow (\psi \rightarrow \varphi) \nonumber \\[-2ex]
	& \mathrm{A}2: (\varphi \rightarrow (\psi \rightarrow \theta)) \longrightarrow ((\varphi \rightarrow \psi) \rightarrow (\varphi \rightarrow \theta)) \nonumber \\[-2ex]
	& \mathrm{A}3: (\neg \varphi \rightarrow \neg \psi) \longrightarrow (\psi \rightarrow \varphi) 
	\end{align}
	\openup -2ex
	\section{Trung bình}
	\openup 2ex
	\begin{gather}
		1_A = 
		\begin{cases}
			1 & \text{if } x \in A \\
			0 & \text{if } x \notin A
		\end{cases} \\	
		n \underbrace{\uparrow \cdots \uparrow}_n = n \rightarrow n \rightarrow n
	\end{gather}
	\begin{align}
		1 \uparrow 1 = {^1}1 &= 1 \nonumber \\
		2 \uparrow \uparrow 2 = {^2}2 &= 4 \nonumber \\
		3 \uparrow \uparrow \uparrow 3 = {^{^{^3}3}}3 % this is so freaking hard
		&= 3 \uparrow \uparrow 3 \uparrow \uparrow 3 = \underbrace{3^{3^{3^{3^{3^{3^{.^{.^{.^{3}}}}}}}}}}_{3^{3^3} \text{threes}}
	\end{align}
	\begin{gather}
		\f{\mathrm{d}}{\mathrm{d} x} f(x) = \lim_{\Delta x \to 0} \f{f(x+\Delta x) - f(x)}{\Delta x} \\
		\mathrm{H_2O}(\ell) + \mathrm{H_2O}(\ell) \rightleftharpoons \mathrm{H_3O^+} (aq) + \mathrm{OH^-} (aq) \\
		\Gamma (n+1) \stackrel{\mathclap{\tiny\mbox{def}}}{=} \int_0^{\infty} e^{-t}t^n \mathrm{d} t \\
		% https://tex.stackexchange.com/questions/74125/how-do-i-put-text-over-symbols
		\gcd(n,m \bmod n); \quad x \equiv y \pmod{b}; \quad x \equiv y \pmod{c}; \quad x \equiv y \quad (d)
	\end{gather}
	\begin{align}
		\nabla \cdot \mathbf{E} &= \f{\rho}{\varepsilon_0} \nonumber \\
		\nabla \cdot \mathbf{B} &= 0 \nonumber \\
		\nabla \times \mathbf{E} &= -\f{\partial \mathbf{B}}{\partial t} \nonumber \\
		\nabla \cdot \mathbf{B} &= \mu_0 \mathbf{J} + \mu_0 \varepsilon_0 \f{\partial \mathbf{E}}{\partial t} \\
		\oiint_{\partial V} \mathbf{E} \cdot \mathrm{d} \mathbf{A} &= \f{Q(V)}{\varepsilon_0} \nonumber \\
		\oiint_{\partial V} \mathbf{B} \cdot \mathrm{d} \mathbf{A}  &= 0 \nonumber \\
		\oint_{\partial S} \mathbf{E} \cdot \mathrm{d} \mathbf{l} &= -\f{\partial \Phi_{B,S}}{\partial t} \nonumber \\
		\oint_{\partial S} \mathbf{B} \cdot \mathrm{d} \mathbf{l} &= \mu_0 I_S + \mu_0 \varepsilon_0 \f{\partial \Phi_{E,S}}{\partial t}
	\end{align}
	\newline
	\begin{gather}
		\rho_\theta
		% https://latex-programming.fandom.com/wiki/Matrix_(LaTeX_environment)
		=
		\begin{pmatrix}
			\cos \theta & \sin \theta \\
			-\sin \theta & \cos \theta 
		\end{pmatrix}
		=
		\begin{bmatrix}
			\cos \theta & \sin \theta \\
			-\sin \theta & \cos \theta 
		\end{bmatrix} \\ 	
		% https://latex-programming.fandom.com/wiki/Array_(LaTeX_environment)
		% https://tex.stackexchange.com/questions/243522/making-thick-horizontal-lines-and-a-thick-vertical-line-in-a-table
		\left[ \begin{array}{c V{1.25} c c c}
			1 & 0 & \cdots & 0 \\
			\hlineB{1.25}
			0 & \ast & \cdots & \ast \\
			\vdots & \vdots & \ddots & \vdots \\
			0 & \ast & \cdots & \ast
		\end{array} \right]
		= 
		\begin{array}{V{1.25} c V{1.25} c c c V{1.25}}
			\hlineB{1.25}
			1 & 0 & \cdots & 0 \\
			\hlineB{1.25}
			0 & \ast & \cdots & \ast \\
			\vdots & \vdots & \ddots & \vdots \\
			0 & \ast & \cdots & \ast \\
			\hline
		\end{array} \\
		\sigma = \sqrt{\f{1}{N} \sum_{i=1}^{N} p_i(x_i-\bar{x})^2} = \sqrt{\f{\displaystyle{\sum_{i=1}^{N} p_i(x_i-\bar{x})^2}}{N}} \\
		% https://www.tutorialspoint.com/tex_commands/atop.htm
		\varphi(n) = n \cdot \prod_{p|n \atop p \text{ prime}} \left( 1 - \f{1}{p} \right)	
	\end{gather}
	Nếu bạn sử dụng \verb+\usepackage{mathtools}+, bạn có thể biến nó thành
	\begin{gather}
		% https://tex.stackexchange.com/questions/153490/atop-vs-substack-for-multiple-lines-under-a-sum
		\varphi(n) = n \cdot \prod_{\substack{p|n \\ p \text{ prime}}} \left( 1 - \f{1}{p} \right) \\
		^4_{12}\mathrm{C}^{5+}_2 \quad 
		^{14}_{\phantom{1}2}\mathrm{C}^{5+}_2 \quad 
		^{\phantom{1}4}_{12}\mathrm{C}^{5+}_2 \quad 
		^{14}\mathrm{C}^{5+}_2 \quad
		_{2}\mathrm{C}^{5+}_2 \\
		\mathbb{Q} \stackrel{\sim}{=} \left\{ \f{a}{b} \Big| a, b \in \mathbb{Z} \text{ and } b \neq 0 \right\} \Big/ \sim \nonumber \\
		\f{a}{b} \sim \f{c}{d} \Longleftrightarrow ad - bc = 0
	\end{gather}
	\begin{align}
		% https://tex.stackexchange.com/questions/424200/what-does-smash-do-and-where-is-it-documented
		% Nén chữ
		1 \uparrow 1 = {^1}1 &= 1 \nonumber \\[-2ex]
		2 \uparrow \uparrow 2 = {^2}2 &= 4 \nonumber \\[-2ex]
		3 \uparrow \uparrow \uparrow 3 = \smash{{^{^{^3}3}}3} % this is so freaking hard
		&= 3 \uparrow \uparrow 3 \uparrow \uparrow 3 = \underbrace{\smash{3^{3^{3^{3^{3^{3^{.^{.^{.^{3}}}}}}}}}}}_{3^{3^3} \text{threes}}
		% https://pandaqitutorials.com/Writing/latex-spacing
		% Xem ở cuối bài
	\end{align}
	\newpage
	\openup -2ex
	\section{Khó}
	%\setstretch{1.0}
	Lệnh \verb+\newcommand{\+\textit{tên lệnh}\verb+}[+\textit{n}\verb+][+\textit{mặc định}\verb+]{+\textit{định nghĩa}\verb+}+ định nghĩa một câu lệnh, với $n$ là
	số tham số và \textit{mặc định} là giá trị mặc định cho tham số đầu tiên.
	Tham số trong ngoặc nhọn \mbox{(\{ {} \})} là bắt buộc, và tham số trong ngoặc vuông ([\phantom{a}]) là tùy ý. 
	Tham số có thể được tham chiếu thông qua \verb|#1|, \verb|#2|, \ldots , \verb|#9|. 
	Sử dụng \verb+\newcommand{\mathset}[2][+\textit{giá trị mặc định cho tham số thứ nhất}\verb+]{+\textit{định nghĩa câu lệnh}\verb+}+, định nghĩa câu lệnh \verb+\mathset+ hiển thị như ở dưới đây. 
	Chú ý vào các phần cụ thể particular, kích cỡ của thanh giữa, kích cỡ của ngoặc nhọn, khoảng cách giữa thanh giữa và các đối tượng xung quanh. \\
	% https://tex.stackexchange.com/questions/23711/strikethrough-text
	\sout{\textbf{Phần này quá khó nên tôi đã quyết định dừng tại đây.}} \\
	\textbf{Update ngày 12/04/2022: Tôi đã tìm được cách giải cho phần này.} \\
	\begin{tabular}{ll}
		\verb|\mathset{1}| & 
		gives $\mathset{1}$ \\
		\verb|\mathset[x]{0\leq x\leq 1}| & 
		gives $\mathset[x]{0 \leq x\leq 1}$ \\
		\verb|\mathset[(x)_i]{|$\displaystyle{\sum_{i} x_i \in A}$\verb|}| & 
		gives $\mathset[(x)_i]{\displaystyle{\sum_{i} x_i \in A}}$ \\
		\verb|\mathset[|$\displaystyle{\sum_{i=1}^{\infty} n^{-s}}$\verb|]{| $n \in A$ \verb|}| & 
		gives $\mathset[\displaystyle{\sum_{i=1}^{\infty} n^{-s}}]{n \in A}$ \\
		\verb|\mathset{}| & gives $\mathset{}$ \\
		% Warning: eye-raping part
		\verb|\mathset[|$\frac{1}{1+\frac{1}{x}}$\verb|]{x \in A}| & 
		gives $\mathset[\frac{1}{1+\frac{1}{x}}]{x \in A}$ \\
		\verb|\mathset[|$\frac{1}{1+\frac{1}{1+\frac{1}{x}}}$\verb|]{x \in A}| & 
		gives $\mathset[\frac{1}{1+\frac{1}{1+\frac{1}{1+\frac{1}{x}}}}]{x \in A}$ \\
		\verb|\mathset[|$\frac{1}{1+\frac{1}{1+\frac{1}{1+\frac{1}{x}}}}$\verb|]{x \in A}| & 
		gives $\mathset[\frac{1}{1+\frac{1}{1+\frac{1}{1+\frac{1}{x}}}}]{x \in A}$ \\
		\verb|\mathset[|$\frac{1}{1+\frac{1}{1+\frac{1}{1+\frac{1}{1+\frac{1}{x}}}}}$\verb|]{x \in A}| & 
		gives $\mathset[\frac{1}{1+\frac{1}{1+\frac{1}{1+\frac{1}{1+\frac{1}{x}}}}}]{x \in A}$ \\
		\verb|\mathset[|$\frac{1}{1+\frac{1}{1+\frac{1}{1+\frac{1}{1+\frac{1}{1+\frac{1}{x}}}}}}$\verb|]{x \in A}| & 
		gives $\mathset[\frac{1}{1+\frac{1}{1+\frac{1}{1+\frac{1}{1+\frac{1}{1+\frac{1}{x}}}}}}]{x \in A}$ \\
		\verb|\mathset[|$\frac{1}{1+\frac{1}{1+\frac{1}{1+\frac{1}{1+\frac{1}{1+\frac{1}{1+\frac{1}{x}}}}}}}$\verb|]{x \in A}| & 
		gives $\mathset[\frac{1}{1+\frac{1}{1+\frac{1}{1+\frac{1}{1+\frac{1}{1+\frac{1}{1+\frac{1}{x}}}}}}}]{x \in A}$ \\
	\end{tabular} \\
	For the 5 last commands, this is the best I can do.
	\section{Insane}
	\section{Diabolical}
\end{document}
