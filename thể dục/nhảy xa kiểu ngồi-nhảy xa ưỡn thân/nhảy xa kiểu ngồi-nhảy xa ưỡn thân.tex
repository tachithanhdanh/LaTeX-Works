\documentclass[12pt,a4paper,notitlepage]{article}
\author{Tạ Chí Thành Danh}
\date{\today}
\title{Bài làm môn thể dục phân môn nhảy xa}
\usepackage[utf8]{vietnam}
\usepackage{amsmath}
\usepackage[left=2cm,right=2cm,top=2cm,bottom=2cm]{geometry}
\begin{document}
\maketitle
\begin{center}
	\textbf{Trình bày các điểm giống nhau và khác nhau giữa nhảy xa kiểu ngồi và kiểu ưỡn thân. 
		Đâu là kiểu nhảy xa được áp dụng phổ biến.}  
\end{center}
\begin{center}
	\textit{Bài làm:}
\end{center}
\begin{itemize}
	\item[$\ast$] Các điểm giống nhau và khác nhau giữa nhảy xa kiểu ngồi và kiểu ưỡn thân:
	\begin{itemize}
		\item[\textbf{-}] \textbf{Giống nhau:} Đều có 3 giai đoạn chạy đà, giậm nhảy, tiếp đất của kiểu “Ngồi”, “Ưỡn thân”.
		\item[\textbf{-}] \textbf{Khác nhau:} Ở giai đoạn trên không: giai đoạn thụ động, người nhảy không thể thay đổi đường bay của trọng tâm cơ thể, nhưng người nhảy có thể sử dụng kỹ thuật các kiểu nhảy xa khác nhau để tận dụng tối đa đường bay của cơ thể trong không gian do giậm nhảy tạo nên.
	\end{itemize}

	\item[$\ast$]Ưu điểm kiểu ưỡn thân so với kiểu ngồi:
	\begin{itemize}
		\item[\textbf{-}] Toàn bộ thân người thành một hình cánh cung, do đó khi gập lại có khả năng vươn xa về trước hơn bình thường so với kiểu ngồi.
		\item[\textbf{-}] Ưỡn thân là 1 trong 2 kiểu nhảy hiện đại được nhiều VĐV cấp cao sử dụng để thi đấu.
	\end{itemize}	
\end{itemize}	
\end{document}